\documentclass[10pt, a4paper]{memoir}

%--------------------------------------------------------- Style setup ---------------------------------------------------------
\aliaspagestyle{cleared}{plain}

%--------------------------------------------------------- Package Setup ---------------------------------------------------------

\usepackage[utf8]{inputenc}
%\usepackage{textcomp} %Use for old MiKTex framework with updates packages
%\usepackage{lmodern} %Modern font
%\usepackage[T1]{fontenc} %T1 Font
\usepackage{amsmath, amsthm, amssymb} %Gives math tools and environments
\usepackage{mathtools} %Extensions to amsmath
\usepackage{mathrsfs} %Provide \mathscr command
\usepackage{nccmath} %Improve space of \displaybreak and more
\usepackage{graphicx} %Include graphics
\usepackage{datetime} %Datetime
\usepackage{advdate} %\today
\usepackage{verbatim} 
\usepackage[english]{babel}
\usepackage{gensymb} %\degree, \celsius, etc.
\usepackage{bm} %\bm
\usepackage{float} %Add H float modifier
\usepackage{wasysym } %Contradiction arrow
\usepackage{enumitem} %Add control to enumerate environment
\usepackage{xpatch}  %\better declaration
\usepackage{capt-of} %\captionof for something not a float
\usepackage{caption}
\usepackage{pdfpages} %Add multipage pdf document option
\usepackage{color}
\usepackage{geometry}
\usepackage[noanswer]{exercise} %Add exercise environment

\usepackage{esdiff} %Add easy diff by \diff
\usepackage{fancyhdr} %Used for building footers and headers
\usepackage{courier} %Courier font
%\usepackage{framed} %Create framed, shaded and leftbar environment
\usepackage{mdframed} %Same as framed just add breakable framed and coloured boxes
%\usepackage{lipsum} % for creating dummy text
\usepackage{cancel} %Used for \cancel that draws a slash through its argument.
\usepackage{collectbox} %\savebox
\usepackage[most]{tcolorbox} %Add coloured boxes
\usepackage{varwidth} %varwidth enviorment is minipage where width is a maximum value
\usepackage{physics}
\usepackage{nameref}%\nameref
\usepackage{siunitx}
\sisetup{exponent-product = \cdot,per-mode=reciprocal-positive-first}
\usepackage{braket} %Add braket notation
\addto\captionsdanish{%
}
\usepackage{hyperref} %Load last ALWAYS
\usepackage{titlesec} %Change space of sections and subsections
\usepackage{cleveref}
\usepackage{pgfplots}


%--------------------------------------------------------- Theorems setup ---------------------------------------------------------
\numberwithin{equation}{section}
\newtheoremstyle{defp}% name
{}% Space above
{}% Space below
{}% Body font
{}% Indent amount1
{\bfseries}% Theorem head font
{\newline}% hPunctuation after theorem head
{0.1em}% Space after theorem head2
{}% Theorem head spec (can be left empty, meaning `normal')

\newtheoremstyle{defq}% name
{1em}% Space above
{}% Space below
{\itshape}% Body font
{}% Indent amount1
{\bfseries}% Theorem head font
{\newline}% hPunctuation after theorem head
{}% Space after theorem head2
{}% Theorem head spec (can be left empty, meaning `normal')

\newtheoremstyle{dotless}%
{}%
{}%
{\itshape}%
{}%
{\bfseries}%
{\newline}%
{1em}%
{}%

\makeatletter
\renewenvironment{proof}[1][\proofname]{\par
    \normalfont \topsep6\p@\@plus6\p@\relax
    \trivlist
    \item[\hskip\labelsep
        \itshape
        #1\@addpunct{.} \newline] }%\ignorespaces}
\makeatother

\renewcommand*{\proofname}{Proof}

\theoremstyle{plain}
\newtheorem{thm}{Theorem}

\theoremstyle{defp}
\newtheorem{de}[thm]{Definition}

\theoremstyle{dotless}
\theoremstyle{definition}
\newtheorem*{eks}{Example}
\theoremstyle{dotless}
\newtheorem{st}[thm]{Theorem}
\theoremstyle{dotless}
\newtheorem{lem}[thm]{Lemma}
\theoremstyle{defp}
\newtheorem{po}{Postulat}
\theoremstyle{defp}
\newtheorem*{ko}{Konstuktion}
\makeatletter
\patchcmd{\th@be}{\thm@headfont{\itshape}}{\thm@headfont{\normalfont}}{}{}
\makeatother
\theoremstyle{be}          % in order to avoid content to be printed in italics
\newtheorem*{be}{Note} 
\theoremstyle{defp}
\newtheorem*{no}{Notation}
\makeatletter
\renewenvironment{proof}[1][\proofname]{\par
  \pushQED{\qed}%
  \normalfont \topsep6\p@\@plus6\p@\relax
  \trivlist
  \item[\hskip\labelsep
        \itshape
%    #1\@addpunct{.}]\ignorespaces% DELETED
    #1]\ignorespaces% ADDED
}{%
  \popQED\endtrivlist\@endpefalse
}
\makeatother

%\usepackage{ucs,babel} %No idea
%\usepackage[all,cmtip]{xy} %No idea


%--------------------------------------------------------- Own Definitions ---------------------------------------------------------
\makeatother %Create double underline
\def\doubleunderline#1{\underline{\underline{#1}}}

\makeatletter %Better arrays
\renewcommand*\env@matrix[1][*\c@MaxMatrixCols c]{%
  \hskip -\arraycolsep
  \let\@ifnextchar\new@ifnextchar
  \array{#1}}
\makeatother

%Create equality symbol with some text variable that get set above the equation
\newcommand\myeq{\mathrel{\overset{\makebox[0pt]{\mbox{\normalfont\tiny\sffamily def}}}{=}}}

\newcommand{\BAR}{% Own bar definition
  \hspace{-\arraycolsep}%
  \strut\vrule % the `\vrule` is as high and deep as a strut
  \hspace{-\arraycolsep}%
}

\newcommand{\ch}{\cosh}
\newcommand{\sh}{\sinh}
\newcommand{\tnh}{\tanh}
\newcommand{\Arcosh}{\operatorname{Arcosh}}
\newcommand{\Arsinh}{\operatorname{Arsinh}}
\newcommand{\Artanh}{\operatorname{Artanh}}
\newcommand{\ord}{\operatorname{ord}}
\newcommand\nm[1]{\left\lVert#1\right\rVert}
\DeclarePairedDelimiter\abss{\lvert}{\rvert}
\renewcommand{\epsilon}{\varepsilon}
\renewcommand{\phi}{\varphi}
\newcommand\lf[1]{\left(#1\right)}
\newcommand\pr[1]{#1^\prime}
\newcommand\lfa[1]{\langle#1\rangle}
\newcommand\lff[1]{\left[#1\right]}
\newcommand\mb[1]{\mathbb{#1}}
\newcommand\mc[1]{\mathcal{#1}}
\newcommand\code[1]{\texttt{#1}}
\newcommand\inproc[1]{\langle#1\rangle}
\DeclarePairedDelimiter\ceil{\lceil}{\rceil}
\DeclarePairedDelimiter\floor{\lfloor}{\rfloor}
\newcommand\ttt[1]{\texttt{#1}}
\newcommand\tsc[1]{\textsc{#1}}
\newcommand{\CS}{C\nolinebreak\hspace{-.05em}\raisebox{.6ex}{\tiny \#}}

%--------------------------------------------------------- Computer Science Setup ---------------------------------------------------------

\usepackage{forest}
\usepackage{adjustbox}
\usepackage{algorithm}
\usepackage[noend]{algpseudocode}

\algrenewcommand{\algorithmicrequire}{\textbf{Input:}}
\algrenewcommand{\algorithmicensure}{\textbf{Output:}}
\algnewcommand\An{\textbf{ And } }
\algnewcommand\Or{\textbf{ Or } }
\algnewcommand\To{\textbf{ to } }

\let\oldReturn\Return
\renewcommand{\Return}{\State\oldReturn}

%for at lave  i align enviorment
\makeatletter
\let\save@measuring@true\measuring@true
\def\measuring@true{%
  \save@measuring@true
  \def\beamer@sortzero##1{\beamer@ifnextcharospec{\beamer@sortzeroread{##1}}{}}%
  \def\beamer@sortzeroread##1<##2>{}%
  \def\beamer@finalnospec{}%
}
\makeatother

%pause efter hvert ligning
\makeatletter
\g@addto@macro\normalsize{%
    \setlength\belowdisplayskip{2pt}
}

\makeatletter
\g@addto@macro\normalsize{%
    \setlength\abovedisplayskip{7pt}
}

%Lille o notation \smallO
\newcommand\smallO{
  \mathchoice
    {{\scriptstyle\mathcal{O}}}% \displaystyle
    {{\scriptstyle\mathcal{O}}}% \textstyle
    {{\scriptscriptstyle\mathcal{O}}}% \scriptstyle
    {\scalebox{.7}{$\scriptscriptstyle\mathcal{O}$}}%\scriptscriptstyle
  }

\usepackage{chngcntr}
\counterwithout{equation}{section} % remove the chapter number
% \counterwithin{equation}{section}  % add the chapter number

\usepackage[newfloat]{minted}

\definecolor{mauve}{HTML}{E0B0FF}
\lstset{
  language=R,
  basicstyle=\footnotesize, 
  numbers=left,
  numberstyle=\tiny\color{gray},
  stepnumber=1,    
  firstnumber=1,
  numbersep=5pt,
  numberfirstline=true,
  tabsize=3,
  frame = single,
  breaklines=true,
  title=\lstname,
  keywordstyle=\color{blue},
  commentstyle=\color{olive},
  stringstyle=\color{mauve}
}

\definecolor{codegray}{gray}{0.85}
\newcommand{\codes}[1]{\colorbox{codegray}{\texttt{#1}}}

\definecolor{block-gray}{gray}{0.85}
\newtcolorbox{myquote}{colback=block-gray,grow to right by=-10mm,grow to left by=-10mm,
boxrule=0pt,boxsep=0pt,breakable}
\makeatletter
\def\quoteparse{\@ifnextchar`{\quotex}{\singlequote}}
\def\quotex#1{\@ifnextchar`{\triplequote\@gobble}{\doublequote}}
\makeatother
\def\singlequote#1`{[StartQ]#1[EndQ]\qOn}
\def\doublequote#1``{[StartQQ]#1[EndQQ]\qOn}
\long\def\triplequote#1```{\begin{myquote}\parskip 1ex#1\end{myquote}\qOn}
\def\qOn{\catcode``=\active}
\def\qOff{\catcode``=12}
\qOn
\def`{\qOff \quoteparse}
\qOff

\usepackage{chngcntr}
\counterwithin{Exercise}{section}



%--------------------------------------------------------- Tikz setup ---------------------------------------------------------

\usetikzlibrary{positioning}


%--------------------------------------------------------- Beginning of document ---------------------------------------------------------
\setlength\parindent{0pt}

\renewcommand{\ExerciseHeader}{\large\textbf{\ExerciseName~\ExerciseHeaderNB} \smallskip\newline}
\renewcommand{\AtBeginExercise}{\hspace{-0.66em}}
\renewcommand{\AnswerHeader}{\large\textbf{\AnswerName~\ExerciseHeaderNB}\smallskip\newline}
\setlength\AnswerSkipBefore{1em}


\setlength\arraycolsep{2 pt}
\setcounter{tocdepth}{2}
\setcounter{secnumdepth}{2}

\titlespacing*{\section}
{0pt}{3ex plus 1ex minus .2ex}{1ex plus .5ex}

\titlespacing*{\subsection}
{0pt}{1ex plus 0.5ex minus .2ex}{1ex plus .2ex}

\openany


\hypersetup{%
    pdfborder = {0 0 0}
}

\usepackage{csquotes}

\usepackage[authordate,backend=bibtex, bibencoding=utf8]{biblatex-chicago}
\addbibresource{ref.bib}

%Styling of quotes
\usepackage{etoolbox}
\AtBeginEnvironment{quote}{\par\singlespacing\small}

\begin{document}

%--------------------------------------------------------- Document Setup ---------------------------------------------------------

\newcommand*\mytitle{\textsc{Software Requirements for ReunionLog}}


\title{\mytitle \\[1ex] \large \textsc{Reunion}\\}
%\date{\AdvanceDate[0]\today \\ }
\date{}

\centering
\maketitle
	{\scshape\LARGE \LaTeX \par}
	\vspace{0.5cm}
	{Lead Programmer and Lead Designer - \Large\scshape Nickgismokato\par}
	\vspace{0.3cm}
	{Design contributor - \Large\scshape Bj\par}
	\vspace{0.3cm}
	{Design contributor \& tester - \Large\scshape Casper\par}
	\vspace{1cm}
	\vfill
	Written by\par
	~Nick \textsc{Laursen}


\thispagestyle{empty}

\newpage

\section*{\centering Preface}

%\raggedright
This is the documentation of the requirements we want to follow when creating the program \textsc{ReunionLog}. We will be following the \textsc{SOLID} principles. This software is a program meant for the guild \textsc{Reunion} in the game \textsc{World Of Warcraft}. This software will use the \ttt{API} from \textsc{WarcraftLogs}. 

\medskip

Most of these requirements have been gathered from multiple months of pre-gathering data from the \textsc{WarcraftLogs} \ttt{API}. This has been done by creating a \textit{proof-of-concept} program with \textsc{Python}.

\medskip

\center

This software is under the \textsc{MIT License}. Read \ttt{LICENSE} for more information.


\tableofcontents

\newpage

\raggedright

\pagestyle{fancy}
\renewcommand{\sectionmark}[1]{\markboth{#1}{}}

\fancyhf{}
\rhead{\fancyplain{}{$ $\leftmark $ $}} % predefined ()
\lhead{\fancyplain{}{$ $\mytitle$ $}} % 1. sectionname, 1.1 subsection name etc
\cfoot{\fancyplain{}{\thepage}}
\fancypagestyle{plain}{%
  \fancyhf{}%
  \fancyfoot[CF]{\thepage}
  \renewcommand{\headrulewidth}{0pt}%
}

%--------------------------------------------------------- Document ---------------------------------------------------------

\chapter{\textsc{WarcraftLogs} and the \ttt{API}}

\section*{Abstract}

This chapter will go through \textsc{WarcraftLogs} and the \ttt{API} correlating. For this chapter we will not go through the actual documentation but rather give a short refer to the documentation and lay out the most important aspect from the site and the \ttt{API}

\section{Website}

The website for \textsc{WarcraftLogs}\footnote{\url{https://www.warcraftlogs.com/}} is a popular website used by guilds to gather data to one single site. This is done though multiple addons and their own in-house software. 

\subsection{Overall form}

The data than can be collected is both in the form of \textsc{Guild} data, \textsc{Raid} data, \textsc{Dungeon} data, \textsc{Character} data and much more. These can be accessed for all through the website. This can be done by everyone anonymously.

\medskip

The website uses \textsc{graphQL} to display most of their data. Both in tables both also in graphs and tables containing graphics. This gives an easier overview for most users. They to also have some options you can choose for the specific data you want to be showed. 

\medskip

A downside to this approach is that a lot of the specific options is not showed. One could reason the "\textit{simpler}" design is because they want all users to use their website, no matter their technical background. 

\subsection{Specific to our needs}

What described in \textbf{Section 1.1.1} sound really great and reasonable useful for most users case. This is indeed the case for most users, but if you want the information not available on the site or you want another way to represent the data, then the site is not for you. This is why we will be using the data given to us by the \ttt{API}.

\medskip

The only need we have of the website is to check if the information we get is also the information displayed on the site.  

\section{Documentation for the \ttt{API}}\label{sec:Doc}

The \ttt{API} and it's documentation is, for a lack of a better word, idiotic made. There exists two \ttt{API}'s. Version 1 and version 2. We will be using the latter. This version has "\textit{better}" documentation than it's counterpart and uses \textsc{OAuth 2.0} for it's \ttt{API} authentication. The documentation can be found two places. For authentication documentation you can find it at \cite{AuthLink} and you can find the actual command call documentation at \cite{DocLink}.

\subsection{\textsc{GraphQL}}

The first thing we should worry about is the authentication. As mentioned in the \textbf{Preface} we have already made successful connection, therefore this will be discussed later.  

\medskip

\textsc{GraphQL} is the schema of how to make calls to the \ttt{API}. This is done by sending you authentication and a "\textit{Query}" call. This is in simple terms just a string with specific data. This data is both used to tell \textsc{WarcraftLogs} server what you would like to receive bout also where in the schema the server would have to lookup this data. Clearly there is a need to create these query strings. 

\medskip

Therefore we will be using \ttt{GraphQL} library from \textsc{Graphql-dotnet}. More information can be found at \cite{GQLdotnet}. If you were to look at \cite{DocLink} you would find there is a lot of commands you can send. Therefore it is important to create an object which can be agile and create all the necessary strings which we will be using.  

\subsection{Limitation of the \ttt{API}}

When it comes to the \ttt{API} a lot of limits start to show. Some in the documentation and some in the actual \ttt{API} itself. Let us start talking about the \ttt{API} itself. When on a free tier, i.e not subscribed to \textsc{WarcraftLogs} website, you can at maximum make 3.600 calls to their server pr. hour. This inherently doesn't sound that bad, but you will later see in depths why this is a bad system. 

\medskip

I will agree that this stops most novice \ttt{DDOS} attacks to their server, but since the introduction of the second version of their \ttt{API}, multiple clients can easily be set up and could potentially help in the \ttt{DDOS} attacks. Let us now explain why this rate limit is bad when you want to make a software that pulls a great amount of data. 

\medskip

Let us say you had a guild that was raiding and you both raided \textsc{Mythic} and \textsc{Heroic} during the same log. Let us furthermore assume you want the data for when they die and how many pulls they were on, but only for the mythic. By scouring through their schema documentation (\cite{DocLink}), you find the \ttt{Report} object inside the documentation. This is essentially what you want. So you create a query string with \ttt{report} as first argument and and then you scour through the documentation and find the \ttt{event} option which deliver a \ttt{ReportEventPaginator} which is just a list event in the log file. This gets returned as data in form of a \ttt{JSON} string. Let us see the documentation for it:

\begin{displayquote}[\cite{DocLink}]
"A set of paginated report events, filterable via arguments like type, source,t arget, ability, etc. This data is not considered frozen, and it can change without notice. Use at your own risk."
\end{displayquote}

This object has an argument you can filter from called \ttt{difficulty}. The documentation for this is as follow:

\begin{displayquote}[\cite{DocLink}]
"\textbf{difficulty}: Optional. Whether or not to filter the fights to a specific difficulty. By default all fights are included."
\end{displayquote}

This is brilliant. We can now filter our \ttt{Event} to mythic only. Later down the line is argue that \ttt{difficulty} is an integer value. Great so now we just need to figure our what integer value. This is the first hurdle we found when trying this. No actual documentation is given for the arguments, except if it is an object or enum type which they have created. So we set up in \textsc{Python} a script to run through with value $n$ where $n\in\set{-1000,-999,\dots,999,1000}$. Then we checked if there where any difference between the \ttt{JSON} strings we received and we got that there was no difference for any $n$ value. 

\medskip

Clearly any user would be confused. Why would an \ttt{API} give you an option and then not document said option and actually do nothing with this option. Your first instinct would now be to contact  \textsc{WarcraftLogs}. So you write them an email and there response they send you back can be seen in \cref{fig:MailSupp}. So you go to their discord find the thread about the \ttt{API} and ask you question. The intention of this documentation is not to out anyone from their support team and therefore I will not share the conversion I had. But trust us when we say that if was probably the most useless information and support we have ever received. 

\medskip

So how did we fix it, you may ask? We looked at the documentation and again read what was written under the \ttt{difficulty} argument. The difficulty argument says it filter fights. So we then took a look at the \ttt{ReportFight} object. We tested this and sure enough it worked. So to simplify what we now have to do if we wish to get the event data only for \textsc{Mythic}. First we need to get all the fight data that had that difficulty i.e the \ttt{fightIDs}. This is done with the \ttt{ReportFlight} object. Then we have to sort our \ttt{ReportEventPaginator} to the \ttt{fightIDs}. Now we have used two calls of our 3.600 rate limit.

\medskip

In our demo in \textsc{Python} it became clear that we also had to pull a lot more data than just fight i.e pulls and so on. We now have to pull every single fiights, every single name from that fight and the event itself. This gives us a rough estimate:

$$l\cdot\lf{a_{\text{Fights}}+a_{\text{Event}}+a_{\text{name}}\cdot p},$$
where $p$ is the amount of players in any fights and $l$ is the amount of different logs we want to pull data from. Since we cannot get \ttt{fightIDs} whilst also getting the event to only get data from the mythics, we would need to do this over two separate calls and also we cannot pull names just from the event log the way we want it, so we have to do it separately. 

\medskip

This is probably the most irritating limit of the \ttt{API} and therefore we would need to make sure our software can handle this limit in case our request limit get met.


\section{Integration from the \ttt{API} to C\ttt{\#}}

As discussed in \ref{sec:Doc} it is clear that we would need many objects to handle everything involving \tsc{WarcraftLogs} \ttt{API} and it's authentication. Clearly we would need objects handling our query strings, \textit{all of them}, handling of the authentication and most importantly handling the data we receive after we have made a call, i.e the response.

\subsection{Authentication}

Creating an authentication for our software is actually quite easy. We don't need any libraries for the creation of the code or the requester. The only thing we need is an users \ttt{client\_id} and \ttt{client\_secret}. The \ttt{tokenURL} is a constant i.e "\textit{https://www.warcraftlogs.com/oauth/token}". Everything else is user input. This is also important but will be dealt with differently.

\medskip

We will need the user to create the \ttt{client\_id} and \ttt{client\_secret} which we will save as local data for further use. This will be done at first time use of the software.

\subsection{Query Strings}

When dealing with query strings within this \ttt{API}, you are in fact dealing with a \ttt{json} call within the \tsc{GraphQL} schema. As noted in \textbf{Section 1.2.1 GraphQL}, we will be using the \ttt{GraphQL} library from \ttt{Graphql-dotnet} (\cite{GQLdotnet}). This library is also under the \tsc{MIT} license, so no issue in license agreements here. It is important to note that \tsc{Facebook} created \tsc{GraphQL}. This library is just an implementation to \tsc{.NET}. You can easily install it by the following command:

\begin{minted}{bash}
$ dotnet add package GraphQL
\end{minted}

This is the only package needed for this library. We don't need serialization or document cashing since we will implement it ourself if needed. 

\subsection{Data extracted from the \ttt{API}}

\newpage

\chapter{Our software main requirements}

\section*{Abstract}

\section{Authentication}

\section{Events}

\section{Query Strings }

\section{\ttt{.CSV} file}

\newpage

\chapter{Flow of \textsc{ReunionLog}}

\section*{Abstract}

\section{Simple Overview}

\section{Needs for each main Requirements}

\newpage

\chapter{Responsibilities for our requirements}

\section*{Abstract}

\section{Responsibilities}

\section{\textsc{UML} Diagram}


\newpage

\appendix
\chapter{Figures}

\section{Pictures}

\begin{figure}[h]
\centering
\includegraphics[width=0.5\textwidth]{mail1.PNG}
\caption{Mail received from the support team at \textsc{WarcraftLogs}}
\label{fig:MailSupp}
\end{figure}


\newpage


\printbibliography
\end{document}

